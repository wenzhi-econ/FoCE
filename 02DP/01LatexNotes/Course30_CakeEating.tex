\documentclass[12pt]{article}

% Packages
\usepackage{setspace,geometry,fancyvrb,rotating}
\usepackage{marginnote,datetime,enumitem}
\usepackage{titlesec,indentfirst}
\usepackage{amsmath,amsfonts,amssymb,amsthm,mathtools}
\usepackage{threeparttable,booktabs,adjustbox}
\usepackage{graphicx,epstopdf,float,soul,subfig}
\usepackage[toc,page]{appendix}
\usdate

% Page Setup
\geometry{scale=0.8}
\titleformat{\paragraph}[runin]{\itshape}{}{}{}[.]
\titlelabel{\thetitle.\;}
\setlength{\parindent}{10pt}
\setlength{\parskip}{10pt}
\usepackage{fourier}    		  % Favourite Font

%% Bibliography
\usepackage{natbib,fancybox,url,color}
\definecolor{MyBlue}{rgb}{0,0.2,0.6}
\definecolor{MyRed}{rgb}{0.4,0,0.1}
\definecolor{MyGreen}{rgb}{0,0.4,0}
\newcommand{\highlightR}[1]{{\emph{\color{MyRed}{#1}}}} 
\newcommand{\highlightB}[1]{{\emph{\color{MyBlue}{#1}}}}
\usepackage[bookmarks=true,bookmarksnumbered=true,colorlinks=true,linkcolor=MyBlue,citecolor=MyRed,filecolor=MyBlue,urlcolor=MyGreen]{hyperref}
\bibliographystyle{econ}

%% Theorem Environment
\theoremstyle{definition}
\newtheorem{assumption}{Assumption}
\newtheorem{definition}{Definition}
\newtheorem{theorem}{Theorem}
\newtheorem{proposition}{Proposition}
\newtheorem{lemma}[theorem]{Lemma}
\newtheorem{example}[theorem]{Example}
\newtheorem{corollary}[theorem]{Corollary}
\usepackage{mathtools}

\begin{document}

%??%??%??%??%??%??%??%??%??%??%??%??%??%??%??%??%??%??%??%??%??%??
%?? title
%??%??%??%??%??%??%??%??%??%??%??%??%??%??%??%??%??%??%??%??%??%??

\title{\bf Cake Eating Problem}
\author{Wenzhi Wang \thanks{This note is written in my pre-doc period at the University of Chicago Booth School of Business.} } 
\date{\today}
\maketitle

\section{Model Setting}

Suppose that you are presented with a cake of size $W_0$ at time $0$. At each period of time $t = 0, 1, \ldots$, you can eat some of the cake but must save the rest. Let $c_t$ be your consumption in period $t$, and let $u(c_t)$ represent the flow of utility from this consumption. Here, we use $u(c) = \log(c)$, which is real valued, differentiable, strictly increasing, and strictly concave, and $\lim_{c \rightarrow 0^+} u^\prime(c) = +\infty$. Therefore, the problem is 
\begin{equation}
    \notag 
    \begin{aligned}
        \max_{\{c_t\}_{t=0}^{\infty}} & \sum_{t=0}^{\infty} \beta^t u(c_t) \\
        \text{ s.t.   } & W_{t+1} = W_t - c_t, \\
        & c_{t}\geq 0,\; W_{t+1} \geq 0, \; t = 0, 1, \ldots. 
    \end{aligned}
\end{equation}

The Bellman equation associated with this problem is 
\begin{equation}
    \notag 
    V(W_t) = \max_{0 \leq c_t \leq W_t} \left\{u(c_t) + \beta V(\underbrace{W_{t+1}}_{=W_t - c_t} )\right\}
\end{equation}


\subsection{Analytical Solution}
We start with a guess that 
\begin{equation}
    \notag 
    V(W) = A + B \log(W),
\end{equation}
where $A$ and $B$ are coefficients to be determined. Given this conjecture, we can write the Bellman equation as 
\begin{equation}
    \notag
    A+B \log (W)=\max _c\left\{\log c+\beta \left(A+B \log (W-c) \right)\right\}.
\end{equation}

The FOC is 
\begin{equation}
    \notag 
    \frac{1}{c}-\frac{\beta B}{W-c}=0 \quad \Rightarrow \quad c=\frac{W}{1+\beta B},\; W-c=\frac{\beta B W}{1+\beta B}.
\end{equation}
Then we have 
\begin{equation}
    \notag 
    \begin{aligned}
    A+B \log (W)= & \log (W)+\log \frac{1}{1+\beta B}+\beta A+\beta B \log (W)+\beta B \log \frac{\beta B}{1+\beta B} \\
    \Rightarrow & \left\{\
    \begin{array}{ll}
        & A=\beta A+\log \frac{1}{1+\beta B}+\beta B \log \frac{\beta B}{1+\beta B} \\
        & B=1+\beta B 
    \end{array}\right.
    \end{aligned}
\end{equation}

After some algebraic calculation, we can obtain the analytic solution to this simple cake-eating problem:
\begin{equation}
    \notag 
    \begin{aligned}
    & c^{\star}(W)=(1-\beta) W \\
    & V(W)=\frac{\log (W)}{1-\beta}+\frac{\log (1-\beta)}{1-\beta}+\frac{\beta \log (\beta)}{(1-\beta)^2}
    \end{aligned}
\end{equation}

\section{Implementation 1}

\subsection{VFI}

Value function iteration means that we start with an arbitrary guess $V_0(W)$. At iteration $i = 1, 2, \ldots$, we compute 

\begin{equation}
    \notag 
    \begin{aligned}
    V_i(W)=T\left(V_{i-1}\right)(W) & =\max _{0 \leq c \leq W}\left\{u(c)+\beta V_{i-1}(W-c)\right\} \\
    c_{i-1}(W) & =\underset{0 \leq c \leq W}{\arg \max }\left\{u(c)+\beta V_{i-1}(W-c)\right\}
    \end{aligned}.
\end{equation}

To put this idea into practice, we first need to {\bf discretize $W$}, that is, to construct a grid of cake-sizes $\overrightarrow{W} \in \mathcal{W} := \{0, \ldots, \overline{W}\}$, and then calculate the following maximization problem in iteration $i$:

\begin{equation}
    \notag
    V_i(\overrightarrow{W})=\max _{0 \leq c \leq \overrightarrow{W}}\left\{u(c)+\beta V_{i-1}(\overrightarrow{W}-c)\right\}.
\end{equation}

However, one important complication is that we need to evaluate $V_{i-1}$ {\bf \emph{off-the-grids}}, that is, there is no guarantee that $\overrightarrow{W}-c \in \mathcal{W}$. Therefore, we need to exploit some interpolation or function approximation techniques.

\subsection{Avoiding interpolation: Solving the maximization problem ``on-the-grid''}

We can re-organize the Bellman equation as 

\begin{equation}
    \notag 
    V_i(\overrightarrow{W})=\max _{0 \leq \overrightarrow{W}^{\prime} \leq \overrightarrow{W}}\left\{u\left(\overrightarrow{W}-\overrightarrow{W}^{\prime}\right)+\beta V_{i-1}\left(\overrightarrow{W}^{\prime}\right)\right\}.
\end{equation}
By choosing $\overrightarrow{W}, \overrightarrow{W}^{\prime} \in \mathcal{W}$, the last-iteration value function $V_{i-1}$ is always evaluated on the grid, thus avoiding interpolation. 


\end{document}
